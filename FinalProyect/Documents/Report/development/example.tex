
    
    \begin{description}
        \item[Examples of direct short citation]    
    \end{description}
    
    Segundo \textcite{spinello2024}, aaa \cite{spinello2024}.
    
    \begin{description}
        \item[Examples of direct long citation]
    \end{description}
    
    Segundo \textcite{spinello2024} \enquote{aaa}. bbb \verb|\enquote{texto}|.
    
    Says that \enquote{ccc} \cite{spinello2024}.
    
    The study of \textcite[p. 107]{rabello2010} ...
    
    The study of \textcite{pargaonkar2021} ...
    
    The studies of \textcites{badgujar2024}{pargaonkar2021} são aplicadas técnicas de ...
    
    The article of \textcite{estevao2023} ...
    
    \textsc{Direct and long citation must be avoided in scientific texts.}
    
    \section{Related Work}
    
    Similar Works ... (quantity -- 5 works);
    
    % Items with markers
    \begin{itemize}
         \item \textbf{Title of article 01 \cite{ogliari2019}}
         
         First paragraph indicates an introduction to the subject...
         
         In the second: what the study sought to analyze, what the objective was...
         
         In the third: what was developed, what application/experiment was carried out...
         
         Last: what conclusions the work reached...
        \newline
        
        \item \textbf{Title of article 02 (Author, year)}
        
         First paragraph indicates an introduction to the subject...
         
         In the second: what the study sought to analyze, what the objective was...
         
         In the third: what was developed, what application/experiment was carried out...
         
         Last: what conclusions the work reached...
        
    \end{itemize}
    
    \section{Materials and Methods}
        Technologies, instruments, and procedures that will be used in the study. Algorithm \ref{alg:id_algo} refers to the Bubblesort sorting method expressed in Python language.
    
        % Source code formatted
        % See: https://en.wikibooks.org/wiki/LaTeX/Source_Code_Listings
        \begin{lstlisting}[  
            %float,
            language=Python, 
            frame=single, 
            numbers=left,
            caption={Bubblesort sorting method},
            label={alg:id_algo} % id for reference
            ]        
        def bubble_sort(alist):
            for i in range(len(alist)-1,0,-1):
                for j in range(i):
                    if alist[i]>alist[i+1]:
                        temp = alist[i]
                        alist[i] = alist[i+1]
                        alist[i+1] = temp    
        \end{lstlisting}
    
        % Figure
        % https://pt.overleaf.com/learn/latex/Inserting_Images
        \begin{figure}[H]
            \centering
            \includegraphics[width=0.5\textwidth]{fig1.jpg} % change the width (range between 0.0 and 1), if the image is not in the correct position
            \caption{My figure}
            \label{fig:id_figura}
        \end{figure}
        
        % Table
        % Online editor: https://www.latex-tables.com
         \begin{table}[ht]
            \centering
            \caption{My table}
            \label{tab:id_tabela}
            \begin{tblr}{
              colspec = {X[l] X[l]}, % Type X defines cells with automatic line break. Use: l=left, r=right, c=center or j=justify for alignment within cells
              width = \linewidth,
              hlines, % includes horizontal border
              vlines, % includes vertical border        
            }
            \textbf{header 1} & \textbf{header 2} \\
            {text to the left} & {There are many variations of Lorem Ipsum passages available, but most have undergone some form of alteration, by injecting humor, or random words that do not even seem credible enough. If you are going to use a passage of Lorem Ipsum, you must be sure that it does not contain anything embarrassing hidden in the middle of the text.}
            \end{tblr}
        \end{table}
    
    \section{Results and Discussion}
    This section should be written in the second part of the work, known as TCC2.
    
    \section{Final Considerations}
    This section should be written in the second part of the work, known as TCC2.
    
    % %%%%%%%%%%%%%%%%%%%%%%%%%%%%%%%%%%%%%%%%%%%%%%%%%%%%%%%%%%%%%%
    % Seção de Referências (gerada automaticamente)
    \printbibliography  % Não remover esta linha
    % %%%%%%%%%%%%%%%%%%%%%%%%%%%%%%%%%%%%%%%%%%%%%%%%%%%%%%%%%%%%%%